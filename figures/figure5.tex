\begin{figure}[t] \centering
    \makebox[0.01\textwidth]{}
    \makebox[0.108\textwidth]{\scriptsize Input 1}
    \makebox[0.108\textwidth]{\scriptsize Input 2}
    \makebox[0.108\textwidth]{\scriptsize Input 3}
    \makebox[0.108\textwidth]{\scriptsize Input 4}
    \\
    \raisebox{0.1\height}{\makebox[0.01\textwidth]{\rotatebox{90}{\makecell{\scriptsize Method A}}}}
    \includegraphics[width=0.108\textwidth]{example-image}
    \includegraphics[width=0.108\textwidth]{example-image}
    \includegraphics[width=0.108\textwidth]{example-image}
    \includegraphics[width=0.108\textwidth]{example-image}
    \\
    \raisebox{0.1\height}{\makebox[0.01\textwidth]{\rotatebox{90}{\makecell{\scriptsize Method B}}}}
    \includegraphics[width=0.108\textwidth]{example-image}
    \includegraphics[width=0.108\textwidth]{example-image}
    \includegraphics[width=0.108\textwidth]{example-image}
    \includegraphics[width=0.108\textwidth]{example-image}
    \\
    \raisebox{0.1\height}{\makebox[0.01\textwidth]{\rotatebox{90}{\makecell{\scriptsize Method C}}}}
    \includegraphics[width=0.108\textwidth]{example-image}
    \includegraphics[width=0.108\textwidth]{example-image}
    \includegraphics[width=0.108\textwidth]{example-image}
    \includegraphics[width=0.108\textwidth]{example-image}
    \\
    \caption{A figure with vertical text for illustration.} 
    \label{fig:teaser}
\end{figure}

